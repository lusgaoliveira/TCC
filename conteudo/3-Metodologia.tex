\chapter{METODOLOGIA} % Descrever o desenho da pesquisa

    Neste capítulo, são apresentados o ambiente e a abordagem metodológica que serão adotados para o 
    desenvolvimento deste estudo, detalhando os procedimentos que serão seguidos desde a coleta e 
    pré-processamento dos dados até a elaboração, treinamento e avaliação do modelo de rede neural.

\section{Coleta dos Dados}

    Para a implementação dos modelos de previsão, será essencial a disponibilização de uma base de dados 
    substancial para o treinamento, validação e teste do modelo, assim como para a realização das inferências 
    sobre a população. No Brasil, o Instituto Nacional de Meteorologia (INMET) é o responsável pelo Banco de 
    Dados Meteorológicos (BDMEP), que tem como finalidade a coleta, armazenamento, processamento e 
    disponibilização de dados e informações sobre variáveis meteorológicas.

    Esses dados poderão ser gerados localmente, por meio de estações meteorológicas convencionais ou 
    automáticas, ou adquiridos remotamente, por sensores orbitais, radares, entre outros 
    dispositivos~\cite{vianna2017}. O Banco de Dados Meteorológicos para Ensino e Pesquisa (BDMEP), 
    especificamente, reunirá informações diárias provenientes das estações da rede do INMET, em 
    conformidade com as normas da Organização Meteorológica Mundial (INMET, s.d.). Os dados utilizados 
    neste estudo corresponderão ao período de 25/06/2009 a 31/12/2024, provenientes da estação meteorológica 
    automática localizada na cidade de Itapetinga-BA.

\section{Pré-processamento dos Dados}

    O pré-processamento dos dados terá início com a análise exploratória, com o objetivo de identificar 
    padrões, anomalias e relações relevantes. De acordo com~\citeonline{tuckey1977}, a análise exploratória 
    de dados (AED) é caracterizada por um processo investigativo, análogo ao trabalho de um detetive, 
    que buscará pistas e evidências, enquanto a análise confirmatória de dados se assemelha ao trabalho 
    judicial, onde as evidências serão testadas e verificadas.

    Com base nesse conceito, será realizado o tratamento de inconsistências nos dados, incluindo a remoção 
    ou correção de valores atípicos (\emph{outliers}), o preenchimento de dados ausentes e a normalização das 
    variáveis para garantir sua compatibilidade com o modelo de rede neural. Essas etapas serão essenciais 
    para garantir que o modelo seja treinado com um conjunto de dados consistente e representativo.

\section{Construção do Modelo de Rede Neural}

    A construção do modelo de rede neural será baseada na arquitetura \emph{Multi-Layer Perceptron} (MLP), 
    uma das mais utilizadas para previsão de séries temporais. Para a implementação do modelo, serão 
    empregadas as bibliotecas \emph{TensorFlow} e \emph{Keras}, que fornecerão suporte para a construção e 
    treinamento de redes neurais. Adicionalmente, serão aplicadas técnicas como \emph{dropout}, 
    regularização L2 e otimização com o algoritmo \emph{Adam} para evitar sobreajuste e aprimorar a 
    capacidade preditiva do modelo.

\section{Treinamento e Validação do Modelo}

    O treinamento do modelo será realizado com o conjunto de dados previamente preparado, utilizando a 
    técnica de validação cruzada e \emph{early stopping} para prevenir sobreajuste. Os hiperparâmetros, 
    como a taxa de aprendizado, o número de camadas ocultas, o número de neurônios por camada e o tamanho 
    do lote (\emph{batch size}), serão ajustados conforme necessário.

    A divisão dos dados será realizada da seguinte forma: 
    \begin{itemize} 
        \item \textbf{Treinamento}: 70\% dos dados, utilizados para ajuste dos pesos da rede neural. 
        \item \textbf{Validação}: 15\% dos dados, empregados para avaliação do desempenho do modelo durante 
        o treinamento e ajuste dos hiperparâmetros. 
        \item \textbf{Teste}: 15\% dos dados, usados para avaliação da capacidade preditiva final do modelo. 
    \end{itemize}

\section{Avaliação e Análise dos Resultados}

    A avaliação do modelo será realizada por meio de métricas amplamente adotadas na previsão de 
    séries temporais, tais como: 
    \begin{itemize} 
        \item \textbf{Erro Quadrático Médio (MSE - Mean Squared Error)}: medida da média dos erros ao quadrado, 
        penalizando desvios maiores.
        \item \textbf{Raiz do Erro Quadrático Médio (RMSE - Root Mean Squared Error)}: fornecendo uma 
        interpretação mais intuitiva do erro, mantendo a mesma unidade da variável predita. 
        \item \textbf{Erro Absoluto Médio (MAE - Mean Absolute Error)}: medida da média dos erros absolutos, 
        sendo menos sensível a grandes \emph{outliers} do que o MSE. 
        \item \textbf{Coeficiente de Determinação (R²)}: medida da proporção da variabilidade explicada 
        pelo modelo, indicando sua qualidade preditiva. 
    \end{itemize}

    A análise dos resultados será baseada na interpretação dessas métricas, permitindo avaliar a precisão 
    do modelo na previsão da temperatura média da cidade de Itapetinga-BA.
