\chapter{METODOLOGIA}
%Descrever o desenho da pesquisa

No decorrer deste capítulo são abordados ambiente e metodologia utilizada durante o
desenvolvimento do presente trabalho.

\section{Obtenção dos Dados}

    Para a aplicação de modelos de previsão, é essencial dispor de uma quantidade significativa de dados para o 
    treinamento, validação e teste do modelo, bem como para a inferência dessas informações sobre a população como um 
    todo. No Brasil, o Instituto Nacional de Meteorologia (INMET) é o órgão responsável pelo Banco de Dados 
    Meteorológicos (BDMEP), planejado para coletar, armazenar, processar e disponibilizar dados e informações sobre 
    variáveis meteorológicas. 

    Esses dados podem ser gerados localmente, por meio de estações meteorológicas convencionais ou automáticas, 
    ou captados remotamente, utilizando sensores orbitais, radares, entre outros dispositivos~\cite{vianna2017}. 
    O Banco de Dados Meteorológicos para Ensino e Pesquisa (BDMEP), em particular, reúne informações meteorológicas 
    diárias provenientes das estações da rede do INMET, seguindo as normas técnicas da Organização Meteorológica 
    Mundial (INMET, s.d.).

    Os dados utilizados no presente trabalho refere-se do período de 25/06/2009 a 31/12/2024, sendo dos dados 
    diários da estação meteorológica automática presente na cidade de Itapetinga-BA.
% O presente trabalhou utilizou os dados da estação meterológica autmática 
\section{Preparação dos Dados}
    Na preparação de dados foi realizada inicialmente uma análise exploratória dos dados, afim de uma inspeção
    inicial dos dados com o objetivo de identificar padrões, anomalias e relações importantes. 
    Tuckey (1977) afirma que a análise exploratória de dados (AED) é um processo investigativo, semelhante ao 
    trabalho de um detetive que busca pistas e evidências, enquanto a análise confirmatória de dados se 
    assemelha ao trabalho judicial, no qual as evidências são analisadas. Dessa forma, foi realizado o tratamento
    de anomalias nos dados, como outliers, dados faltantes entre outros. CITAR

% \section{Desenvolvimento do Modelo de Rede Neural}
%     Falar sobre como desenvolveu o modelo, as bibiliotecas que serão utilizadas para a construção do mesmo
%     e possíveis técnicas.

% \section{Treinamento e Validação do Modelo}
%     Falar sobre os hiperparametros do modelo e técnica de validação
% \section{Avaliação e Análise dos Resultados}
%     Falar dos resultados7777

\section{Aplicação de Redes Neurais Artificiais}
    Para desenvolver o modelo foi utilizado o ambiente do \emph{Google Colab} sendo \emph{python} a linguagem
    utilizada. Além disso, os dados foram divididos em três grupos: o primeiro para ser utilizado na fase de 
    treinamento, o segundo para a fase de validação e o último para teste do modelo.
    