\chapter{METODOLOGIA}
%Descrever o desenho da pesquisa

No decorrer deste capítulo são abordados ambiente e metodologia utilizada durante o
desenvolvimento do presente trabalho.

\section{Obtenção dos Dados}

Para a aplicação de modelos de previsão, é essencial dispor de uma quantidade significativa de dados para o 
treinamento, validação e teste do modelo, bem como para a inferência dessas informações sobre a população como um 
todo. No Brasil, o Instituto Nacional de Meteorologia (INMET) é o órgão responsável pelo Banco de Dados 
Meteorológicos (BDMEP), planejado para coletar, armazenar, processar e disponibilizar dados e informações sobre 
variáveis meteorológicas. 

Esses dados podem ser gerados localmente, por meio de estações meteorológicas convencionais ou automáticas, 
ou captados remotamente, utilizando sensores orbitais, radares, entre outros dispositivos~\cite{vianna2017}. 
O Banco de Dados Meteorológicos para Ensino e Pesquisa (BDMEP), em particular, reúne informações meteorológicas 
diárias provenientes das estações da rede do INMET, seguindo as normas técnicas da Organização Meteorológica 
Mundial (INMET, s.d.).

% O presente trabalhou utilizou os dados da estação meterológica autmática 
\section{Preparação dos Dados}