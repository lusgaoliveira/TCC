\chapter{INTRODUÇÃO}

Aqui é onde ficara a introdução da problemática

\section{OBJETIVOS}
    O presente projeto tem como objetivo utilizar redes neurais na análise e previsão das variações meteorológicas no 
    município de Itapetinga-BA ao longo do tempo, identificando padrões sazonais, tendências de longo prazo e 
    possíveis anomalias climáticas, a fim de contribuir com o planejamento urbano, agrícola e ambiental da região.
    
    Este trabalho também contará com os seguintes objetivos específicos:

    \begin{itemize}
        %\setlength{\itemsep}{1em}
        \item Analisar as séries temporais dos dados de temperatura, precipitação e outros 
              parâmetros climáticos no município de Itapetinga-BA, utilizando redes neurais artificiais.
        \item Identificar tendências de aquecimento, variações de precipitação e outros 
              impactos ambientais na cidade, por meio de modelos de previsão baseados em redes neurais.
        \item Avaliar as mudanças climáticas em Itapetinga-BA, comparando as previsões 
              geradas pelas redes neurais com os dados históricos para identificar anomalias.
    \end{itemize}

\section{JUSTIFICATIVA}
    %Talvez alterar e não falar do código florestl no momento, começar falando do mundo, mudanças climáticas,brasil, bahia e previsão com redes neurais
    Suspeita-se que as mudanças ambientais e climáticas tenham como principal responsável a ação humana, 
    impulsionada pela intensa atividade industrial. A revolução industrial marcou o início dessa transformação, 
    promovendo a adoção de novas fontes de energia e o fortalecimento do consumo de combustíveis fósseis, como o 
    carvão mineral inicialmente e, posteriormente, o petróleo~\cite{mendoncca2006aquecimento}.
    
    À vista disso, nas últimas décadas, os debates sobre as mudanças climáticas e a necessidade de uma sociedade 
    mais consciente e participativa na preservação ambiental e no desenvolvimento sustentável se tornaram cada 
    vez mais intensos. Em 1972, ao identificar a vulnerabilidade do planeta Terra, houve um esforço mundial conjunto
    em entender os problemas ambientais e ponderar medidas para previnir ou amenizar determinados crises.
    
    A Conferência das Nações Unidas para o Meio Ambiente Humano, também conhecida como Conferência de Estocolmo, 
    realizada em 1972 na cidade de Estocolmo, na Suécia, foi um marco histórico por ser a primeira conferência 
    global com foco no meio ambiente. Durante o evento, deu-se início à estruturação de mecanismos de proteção 
    ambiental, que foram ampliados na Segunda Conferência das Nações Unidas sobre o Meio Ambiente e 
    Desenvolvimento, realizada em 1992, conhecida como Rio-92. Nessas conferências, foram estabelecidos 
    diversos acordos para a proteção do meio ambiente, da biodiversidade e de outros aspectos relacionados à 
    sustentabilidade, como a Agenda 21~\cite{passos2009}.


    Entretanto, de acordo com relatório especial publicado em 2020 pelo Painel Intergovernamental sobre Mudanças Climáticas (IPCC), 
    desde o período pré-industrial, a temperatura média do ar na superfície da Terra quase dobrou em relação à média 
    global registrada anteriormente. Além disso, estima-se que 23\% das emissões antrópicas de gases de efeito estufa 
    sejam provenientes de atividades relacionadas à agricultura, silvicultura e outras práticas agrícolas.

    No Brasil, em 2019, logo após o início do mandato do governo eleito no ano anterior, as invasões às terras 
    indígenas por grupos ilegais, como garimpeiros, foram retomadas. Como resultado, a taxa de desmatamento em 
    junho daquele ano já apresentava um aumento alarmante de 60\% em relação ao mesmo mês do ano anterior. 
    Além disso, houve uma intensificação de atividades ilícitas, como a grilagem de terras, mineração clandestina 
    e exploração madeireira na Amazônia~\cite{barretto2020}.

    \citeonline{barretto2020} afirma que esse aumento foi impulsionado pelos questionamentos levantados pelo então 
    presidente da República, Jair Messias Bolsonaro, que colocava em dúvida a veracidade das informações fornecidas 
    por órgãos públicos responsáveis pelo monitoramento ambiental. Além disso, suas declarações sobre a desarticulação do sistema de 
    regulação ambiental reforçaram, entre seus apoiadores, a percepção de que havia uma espécie de "liberação total".
    Isso resultou na intensificação de práticas prejudiciais ao meio ambiente, à saúde pública e ao tecido social.

    %Falar da bahia e suas mudanças climáticas se possível
    %Elencar com a utilização de métodos de previsão para influenciar novas políticas públicas


\section{ORGANIZAÇÃO DOS CAPÍTULOS}
%Falar sobre como se dará os capítulos, quais são e o que cada um irá abordar.