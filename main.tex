\documentclass[12pt, openright, a4paper, brazil, oneside]{abntex2}

\usepackage{lmodern}                 
\usepackage[T1]{fontenc}             
\usepackage[utf8]{inputenc}         
\usepackage[brazil]{babel}           
\usepackage{amsmath, amsfonts, amssymb, amsthm} 
\usepackage{graphicx}                
\usepackage{microtype}              
\usepackage{indentfirst}             
\usepackage{color}                   
\usepackage{hyperref}                
\usepackage[top=3cm,left=3cm,right=2cm,bottom=2cm]{geometry}
\graphicspath{{figuras/}}

% Configurações de aparência do PDF final
\hypersetup{
    pdftitle={Utilização de Redes Neurais na Análise e Previsão de Séries Temporais Meteorológicas em Itapetinga-BA}, 
    pdfauthor={Lucas Silva de Oliveira},
    pdfsubject={Trabalho de Conclusão de Curso},
    pdfcreator={LaTeX with abnTeX2},
    pdfkeywords={redes neurais}{análise de séries temporais}{meteorologia},
    colorlinks=true,                
    linkcolor=black,                 
    citecolor=black,                
    filecolor=magenta,             
    urlcolor=black,
    breaklinks=true
}

\bibliographystyle{abntex2-alf}

% Documento
\begin{document}

    % Capa
    \begin{capa}
        \center
        \includegraphics[width=2cm]{template/ifbaiano.png}\\[0.2cm]
        {\ABNTEXchapterfont\large INSTITUTO FEDERAL DE EDUCAÇÃO, CIÊNCIA E TECNOLOGIA BAIANO}\\[0.2cm]
        {\ABNTEXchapterfont\large Bacharelado em Sistemas de Informação} \\[4cm]

        {\ABNTEXchapterfont\bfseries\LARGE UTILIZAÇÃO DE REDES NEURAIS NA ANÁLISE E PREVISÃO DE SÉRIES TEMPORAIS METEOROLÓGICAS EM ITAPETINGA-BA}\\[4cm]

        {\ABNTEXchapterfont\large Lucas Silva de Oliveira}\\[4cm]

        \vfill
        \large Itapetinga - Bahia\\
        \today
    \end{capa}

    % Folha de rosto
    \begin{folhaderosto}

        \begin{center}
            \ABNTEXchapterfont\bfseries\LARGE UTILIZAÇÃO DE REDES NEURAIS NA ANÁLISE E PREVISÃO DE SÉRIES TEMPORAIS METEOROLÓGICAS EM ITAPETINGA-BA
        \end{center}

        \vspace{4cm}

        \begin{center}
            {\ABNTEXchapterfont\large Lucas Silva de Oliveira}
        \end{center}

        \vspace{4cm}

        \vspace*{\fill}
        \begin{flushright}
            \begin{minipage}{8.6cm}
                Trabalho de Conclusão de Curso apresentado como 
                requisito parcial para obtenção 
                do título de Bacharel em Sistemas de Informação.
                
                \vspace{0.5cm}
                \textbf{Orientador(a)}: Prof(a). Dr(a). Nome do(a) Orientador(a).
            \end{minipage}
        \end{flushright}

        \vspace{5.5cm}

        \begin{center}
            \large Itapetinga - Bahia\\
            \today
        \end{center}

    \end{folhaderosto}


    % Dedicatória
    \begin{dedicatoria}
        \vspace*{\fill}
        \begin{flushright}
            \noindent
            \textit{Dedico este trabalho a\ldots}
        \end{flushright}
    \end{dedicatoria}

    % Agradecimentos
    \begin{agradecimentos}
        Agradeço a\ldots
    \end{agradecimentos}

    % Epígrafe
    \begin{epigrafe}
        \vspace*{\fill}
        \begin{flushright}
            \textit{
            "Coloca aqui a epígrafe."\\
            --- Nome do autor da epígrafe}
        \end{flushright}
    \end{epigrafe}

    % Resumo
    \begin{resumo}
        Aqui fica o resumo em português.\\   
        \textbf{Palavras-chave}: redes neurais, séries temporais, meteorologia.
    \end{resumo}

    % Abstract
    \begin{abstract}
        Aqui fica o resumo em inglês.\\ 
        \textbf{Keywords}: neural networks, time series, meteorology.
    \end{abstract}

    \newpage

    \tableofcontents
 
    % Corpo do Trabalho

    \chapter{INTRODUÇÃO}

Aqui é onde ficara a introdução da problemática

\section{OBJETIVOS}
    Analisar e prever variações da temperatura média no município de Itapetinga-BA ao longo do tempo
    utilizando redes neurais artificiais.

\section{OBJETIVOS ESPECÍFICOS}
    \begin{itemize}
        %\setlength{\itemsep}{1em}
        \item Analisar séries temporais de temperatura média em Itapetinga-BA, utilizando redes neurais 
        do tipo MLP.
        \item Identificar tendências de aquecimento e outros impactos ambientais, empregando um modelo 
        de previsão baseado em aprendizado profundo.
        \item Avaliar o desempenho do modelo MLP considerando diferentes configurações, como número de 
        camadas ocultas, neurônios por camada e funções de ativação.
    \end{itemize}

\section{JUSTIFICATIVA}
    %Talvez alterar e não falar do código florestl no momento, começar falando do mundo, mudanças climáticas,brasil, bahia e previsão com redes neurais
    Suspeita-se que as mudanças ambientais e climáticas tenham como principal responsável a ação humana, 
    impulsionada pela intensa atividade industrial. A revolução industrial marcou o início dessa transformação, 
    promovendo a adoção de novas fontes de energia e o fortalecimento do consumo de combustíveis fósseis, como o 
    carvão mineral inicialmente e, posteriormente, o petróleo~\cite{mendoncca2006aquecimento}.
    
    À vista disso, nas últimas décadas, os debates sobre as mudanças climáticas e a necessidade de uma sociedade 
    mais consciente e participativa na preservação ambiental e no desenvolvimento sustentável se tornaram cada 
    vez mais intensos. Em 1972, ao identificar a vulnerabilidade do planeta Terra, houve um esforço mundial conjunto
    em entender os problemas ambientais e ponderar medidas para previnir e amenizar determinados crises.
    
    A Conferência das Nações Unidas para o Meio Ambiente Humano, também conhecida como Conferência de Estocolmo, 
    realizada em 1972 na cidade de Estocolmo, na Suécia, foi um marco histórico por ser a primeira conferência 
    global com foco no meio ambiente. Durante o evento, deu-se início à estruturação de mecanismos de proteção 
    ambiental, que foram ampliados na Segunda Conferência das Nações Unidas sobre o Meio Ambiente e 
    Desenvolvimento, realizada em 1992, conhecida como Rio-92. Nessas conferências, foram estabelecidos 
    diversos acordos para a proteção do meio ambiente, da biodiversidade e de outros aspectos relacionados à 
    sustentabilidade, como a Agenda 21~\cite{passos2009}.

    Entretanto, de acordo com relatório especial publicado em 2020 pelo Painel Intergovernamental sobre Mudanças Climáticas (IPCC), 
    desde o período pré-industrial, a temperatura média do ar na superfície da Terra quase dobrou em relação à média 
    global registrada anteriormente. Além disso, estima-se que 23\% das emissões antrópicas de gases de efeito estufa 
    sejam provenientes de atividades relacionadas à agricultura, silvicultura e outras práticas agrícolas.

    No Brasil, em 2019, logo após a transição para a nova gestão federal, as invasões às terras 
    indígenas por grupos ilegais, como garimpeiros, foram retomadas. Como resultado, a taxa de desmatamento em 
    junho daquele ano já apresentava um aumento alarmante de 60\% em relação ao mesmo mês do ano anterior. 
    Além disso, houve uma intensificação de atividades ilícitas, como a grilagem de terras, mineração clandestina 
    e exploração madeireira na Amazônia~\cite{barretto2020}.

    \citeonline{barretto2020} afirma que esse aumento foi impulsionado por questionamentos feitos pelas autoridades governamentais 
    naquele momento, que duvidavam da veracidade das informações fornecidas pelos órgãos responsáveis pelo monitoramento ambiental. 
    Além disso, declarações sobre a possível flexibilização da regulamentação ambiental reforçaram, entre certos grupos, a percepção 
    de uma "liberação total". Isso resultou na intensificação de práticas prejudiciais ao meio ambiente, à saúde pública e ao tecido 
    social.

    Segundo dados do MapBiomas, o Brasil já havia perdido cerca de 20\% de suas áreas naturais até 1985. Entre 
    1985 e 2023, essa perda se intensificou, aumentando em mais 13\%, atingindo um total de 33\% do território 
    nacional. A velocidade alarmante dessa transformação na cobertura e no uso do solo contribui significativamente 
    para o agravamento dos riscos climáticos no país. No ano de 2023, a Bahia se destacou como o segundo estado 
    com maior taxa de desmatamento. Em comparação com 2022, houve um aumento de 27\% na área desmatada, sendo o 
    Cerrado o bioma mais afetado, respondendo por 67\% do total. Na sequência, aparecem a Caatinga e a Mata 
    Atlântica como os biomas mais impactados pelo desmatamento no estado~\cite{polcri2024}.
    
    
    Dessa forma, torna-se fundamental dispor de instrumentos capazes de prever eventos climáticos com baixa margem de erro 
    e antecedência suficiente para viabilizar a construção de soluções e estratégias eficazes na mitigação de danos. 
    Nesse contexto, destacam-se os modelos de \emph{machine learning}, que, ao utilizar dados históricos meteorológicos, podem 
    realizar predições mais assertivas e robustas para auxiliar na tomada de decisão, permitindo um 
    planejamento mais eficiente em setores como agricultura, energia e gestão de desastres naturais.
    
    Nesse sentido, este trabalho tem como foco analisar as mudanças na temperatura média no município de Itapetinga-BA, 
    cidade historicamente conhecida como a 'Capital da Pecuária'. Além de contar com um setor industrial 
    significativo, Itapetinga abriga um \textit{campus} avançado da Universidade Estadual do Sudoeste da Bahia e um campus 
    do Instituto Federal Baiano, consolidando-se como um importante polo educacional e econômico na região.

% \section{ORGANIZAÇÃO DOS CAPÍTULOS}
% %Falar sobre como se dará os capítulos, quais são e o que cada um irá abordar.
%     Este trabalho está dividido da seguinte forma: No capítulo 2, é apresentado uma visão geral dos conceitos teóricos que 
%     fundamentam este projeto, como séries temporais, suas características e modelos de previsão. Além disso, é introduzido o 
%     conceito de redes neurais, abordando seus componentes, aprendizado, métricas e a arquitetura utilizada neste trabalho. O capítulo 3 
%     detalha os métodos e técnicas utilizados na pesquisa. No capítulo 4, são mostrados os resultados obtidos. Por fim, no 
%     capítulo 5, são apresentamos as considerações finais, destacando as principais contribuições do trabalho e propondo direções 
%     para futuras pesquisas.

    \chapter{REFERENCIAL TEÓRICO}
Para o entendimento e progresso deste trabalho, faz-se necessária a compreensão de conceitos relacionados a 
Séries Temporais, incluindo suas técnicas e modelagem, fontes de Dados Meteorológicos, 
aplicações de séries temporais em Meteorologia, redes neurais e séries temporais com redes neurais. 

\section{Séries Temporais}
    Muitas pessoas, em algum momento, já imaginaram como seria prever o futuro e ter acesso a informações sobre eventos 
    ou situações de suas vidas. Essa curiosidade reflete um desejo universal, mas também uma necessidade presente em 
    diversas áreas, como na gestão governamental, no setor financeiro e em contextos sociais. Nesse cenário, surge o 
    conceito de Série Temporal, definido como um conjunto de observações organizadas sequencialmente no tempo, 
    representadas por \( x_t \), com cada valor correspondente a um instante específico \(t\) \cite{box2015}. O estudo de 
    Séries Temporais permite não apenas compreender as características de fenômenos que evoluem ao longo do tempo, mas 
    também desenvolver e ajustar modelos estatísticos capazes de explicar ou prever o comportamento dos dados 
    observados.
    
    De acordo com~\cite{brockwell2002}, séries temporais podem ser classificadas 
    discretas e continuas, uma série temporal é discreta quando o conjunto \( t_0 \) de tempos em que as observações 
    são feitas é um conjunto discreto, como o caso de observações que são realizadas em um determinado intervalo de 
    tempo fixo. Sendo denotada por:
    \begin{equation}
        \{X_t : t \in T\}, \quad T = \{t_1, \dots, t_n\}
    \end{equation}
    
    
    
    E seríes temporais continuas quando suas observações são obtidas continualmente 
    no tempo. 
    \begin{equation}
        \{X(t) : t \in T\}, \quad T = \{t : t_1 < t < t_2\}
    \end{equation}
        

    
    Ao iniciar a análise de uma série temporal é de alta valia utilizar de gráficos criados sequencialmente no tempo,
    visto que isso pode revelar determinados padrões de comportamento e algumas características que podem estar 
    presentes nos dados, como tendência, sazonalidade, ciclidade e ruído também chamado de erro aleatório~\cite{costa2019}. 

    \subsection{Decomposição}

        A tendência ($\mu_t$) é falar o que é a tendência \\
        
        A ciclidade ($\psi_t$) pode ser \\
        
        A sazonalidade ($\gamma_t$) pode ser \\
        
        O ruído ($\epsilon_t$) é 

        %mostrar fórmula



\section{Redes Neurais}
%Falar brevemente
    O cérebro humano é um computador de grande complexidade, não linear e paralelo, 
    composto por cerca de 10 bilhões de neurônios, cada um conectado com outros 10 
    bilhões de neurônios. Em sua composição existe o corpo da célula também chamado de 
    soma, e canais da saída e entrada (dendritos e axônios) que conectam os neurônios, 
    dendritos também são chamadas zonas receptivas e axônios de linhas de transmissão. 
    Cada neurônico recebe informações eletroquimicas de outros neurônios nos dendritos 
    através dos axônios. Se as somas dessas entradas elétricas for suficientemente forte 
    para ativar o neurônio, um sinal eletroquímico ao longo do axônio, possuindo a 
    habilidade de organizar os neurônios, um dos seus componentes integradores, para 
    assim realizar várias formas de processamento de forma mais rápida que os 
    computadores digitivas convencionais~\cite{haykin2009neural}.

    A Figura \ref{fig:celula_piramidial} mostra a estrutura do neurônio:
    \begin{figure}[!htb]
        \centering
        \caption{Célula piramidial.}
        \includegraphics[scale=0.5]{celula-piramidial.png}\\
        {\footnotesize Fonte: The University Of Queensland.}\
        \label{fig:celula_piramidial}
    \end{figure}
        
    \subsection{Histórico}
        As redes neurais são frequentemente consideradas um complemento à computação tradicional. Curiosamente, 
        John von Neumann, amplamente reconhecido como o pai da computação moderna devido à sua proposta da arquitetura 
        que possibilitou a criação do computador de programa armazenado, demonstrava grande interesse em modelar o 
        funcionamento do cérebro humano. Esse interesse levantou debates entre pesquisadores sobre a possível interação 
        entre as ideias de von Neumann e os primórdios das redes neurais. Alguns estudiosos destacam indícios que 
        sugerem a visão de von Neumann sobre as direções futuras do desenvolvimento dos computadores~\cite{Fausett1994}.

        \subsubsection{Perceptrons}
            
            Em 1958, o psicólogo Frank Rosenblatt publicou um artigo que, pela primeira vez, descreveu de forma 
            algorítmica o funcionamento de um modelo de rede neural para aprendizagem supervisioanda. Essa 
            publicação inspirou inúmeros pesquisadores a direcionarem seus esforços para estudos sobre redes neurais, 
            explorando diversos aspectos dessa temática ao longo das décadas de 1960 e 1970~\cite{haykin2009neural}.

            \begin{figure}[!htb]
                \centering
                \caption{Fluxo do perceptron.}
                \includegraphics[scale=0.8]{fluxo-perceptron.png}\\
                {\footnotesize Fonte: Haykin (2009).}\
                \label{fig:fluxo-perceptron}
            \end{figure}

            Como apresentado na Figura~\ref{fig:fluxo-perceptron}, o perceptron consiste de um único neurônio com 
            pesos sinápticos ajustáveis e um viés. Ele possui uma camada de entrada (a retina) conectada aos pesos e 
            uma camada de saída. Seu funcionamento baseia-se em um combinador linear seguido por uma função de 
            ativação que realiza uma função linear. Esse nó somador (o neurônio) calcula uma combinação linear das 
            entradas aplicadas às suas sinapses, além de incorporar um viés aplicado externamente que ajusta a posição
            da função de ativação. O resultado dessa soma é passado à função de ativação, que produz uma saída de +1 
            se a entrada for positiva, ou -1, se for negativa. O perceptron é um classificador binário, pois resolve 
            apenas problemas de classificação de padrões linearmente separáveis, ou seja, é capaz de lidar 
            exclusivamente com problemas nos quais duas classes podem ser separadas por uma 
            linha em um hiperplano~\cite{haykin2009neural}.


        \subsubsection{Adaline}

            Em 1960, Bernard Widrow e Marcian Hoff desenvolveram uma regra de aprendizagem denominada "Regra Delta", 
            também conhecida como Least Mean Squares (LMS) ou método do Gradiente Descendente. Com base nessa regra, 
            foi criada uma rede neural com a mesma estrutura do Perceptron, composta por uma camada de entrada, uma 
            camada de saída e um único neurônio. A diferença principal residia na regra de aprendizado empregada para 
            o ajuste dos pesos.
            
            A Regra Delta, que tem como finalidade ajustar os pesos do neurônio, busca minimizar a diferença entre a 
            saída desejada e a resposta obtida a partir da combinação linear de todas as amostras. Utilizando a 
            minimização do erro quadrático médio entre os valores previstos e reais, o método opera dentro de um 
            contexto de aprendizagem supervisionada, onde há uma saída esperada previamente definida. O algoritmo 
            ajusta iterativamente o vetor de pesos \( w\) atribuído à rede, com o objetivo de determinar um 
            \( w^{*} \) ótimo tal que o erro quadrático \({E(w{*})}\), calculado sobre todo o conjunto de amostras, 
            seja minimizado.

            Essa rede neural foi projetada para aplicações em sistemas de chaveamento de circuitos telefônicos e 
            ficou conhecida como Adaline (Adaptive Linear Neuron). A Adaline foi uma das primeiras redes neurais 
            implementadas em contextos industriais, marcando um avanço significativo na aplicação de tecnologias 
            baseadas em inteligência artificial. Além disso, a regra de aprendizagem Widrow-Hoff para uma rede neural 
            de apenas uma camada foi o percursor da regra de Backpropagation para múltiplas camadas~\cite{Fausett1994,silva2010}
        \subsubsection{Backpropagation}
      
            FALAR COMO A BACKPROPAGATION FEZ AS REDES NEURAIS EVOLUIREM E VOLTAREM A SEREM ESTUDADAS
    \subsection{Componentes das Redes Neurais}
        %

    \subsection{Aprendizado em Redes Neurais}
    \subsection{Arquitetura}
    %Falar o que é e botar uma imagem em todas
        \subsubsection{Feed-Forward}
        \subsubsection{Multi Layer Perceptron}
        \subsubsection{Redes Neurais Recorrentes}
        \subsubsection{Redes Neurais Convolucionais}
        \subsubsection{Long Short-Term Memory (LSTM)}


%Talvez por na metodologia
\section{Dados Meteorológicos}

    Para a aplicação de modelos de previsão, é essencial dispor de uma quantidade significativa de dados para o 
    treinamento, validação e teste do modelo, bem como para a inferência dessas informações sobre a população como um 
    todo. No Brasil, o Instituto Nacional de Meteorologia (INMET) é o órgão responsável pelo Banco de Dados 
    Meteorológicos (BDMEP), planejado para coletar, armazenar, processar e disponibilizar dados e informações sobre 
    variáveis meteorológicas. 
    
    Esses dados podem ser gerados localmente, por meio de estações meteorológicas convencionais ou automáticas, 
    ou captados remotamente, utilizando sensores orbitais, radares, entre outros dispositivos~\cite{vianna2017}. 
    O Banco de Dados Meteorológicos para Ensino e Pesquisa (BDMEP), em particular, reúne informações meteorológicas 
    diárias provenientes das estações da rede do INMET, seguindo as normas técnicas da Organização Meteorológica 
    Mundial (INMET, s.d.).

    \chapter{METODOLOGIA}
%Descrever o desenho da pesquisa

\section{Preparação dos Dados}

    \chapter{RESULTADOS ESPERADOS}
    Espera-se que o modelo baseado em redes neurais artificiais consiga prever as oscilações na 
    temperatura média do município de Itapetinga-BA com um nível satisfatório de precisão. Além disso, 
    busca-se identificar padrões na série temporal estudada, proporcionando uma compreensão mais detalhada 
    das dinâmicas climáticas locais. Os resultados obtidos deste trabalho poderão contribuir para pesquisas 
    futuras sobre as alterações do clima na região.
    \include{conteudo/5-Consideracoes}

    % Referências
    \bibliography{referencias}

\end{document}
